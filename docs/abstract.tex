\chapter*{Abstract}
\thispagestyle{empty}
This paper presents a comparative analysis of functional programming paradigms in Go and F\#, focusing on key concepts such as Algebraic Data Types and First-class Functions. The \textit{NerdDeck} Flash Card application serves as a practical example to illustrate the application of functional programming principles in both languages. Despite inherent challenges arising from the impurity of both languages, the comparison reveals distinctions in their approaches, emphasizing F\#'s native support for functional programming and Go's adaptation within its multi-paradigm framework.

The challenges encountered underscore the importance of balancing functional and imperative programming in each language, prompting a nuanced approach in selecting languages for projects prioritizing functional programming principles. Looking ahead, further exploration into writing purer functional code within projects like \textit{NerdDeck} is proposed. This could involve delving deeper into functional programming concepts and considering the use of purely functional languages like Haskell. As programming languages and paradigms continue to evolve, ongoing advancements may present new opportunities and challenges for functional programming practitioners.

The insights gained from this comparative analysis contribute to a broader understanding of functional programming in diverse language ecosystems and provide a foundation for future exploration and research in this evolving field.

\bigskip

\noindent
Keywords: Functional Programming, Go Programming Language, F\# Programming Language, Software Engineering

